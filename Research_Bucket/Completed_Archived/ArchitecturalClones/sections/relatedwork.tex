\vspace{-10pt}
\section{Related Work}
\label{sec: relatedwork}
While no previous works have investigated architectural tactics as we have, numerous previous studies have analyzed code clones and their impact on software development. Juergens et al.\cite{juergens2009code} studied the consequences that code clones had on program correctness. This work found that commercial and open source software systems often suffer from inconsistent changes due to the presence of code clones, thus leading to possible system faults and increased maintenance costs. Many previous works have stated that code clones are undesirable since they often lead to more bugs and make their remediation process more difficult and expensive~\cite{Mondal:2012:ESC:2387358.2387360,Duala-Ekoko:2010:CRD:1767751.1767754}. Other research has shown that clones may also substantially raise the maintenance costs associated with an application~\cite{juergens2009code}, the importance of which is highlighted by the fact that the maintenance phase of a project has been found to encompass between 40\% and 90\% of the total cost of a software project~\cite{Shukla:2008:ESM:1342211.1342232}. Ultimately, unintentionally making inconsistently applied bug fixes to cloned code across a software system increases the likeliness of further system faults~\cite{Deissenboeck_2010}.


%%% Mention some other clone detection techniques
Nicad is a powerful text-based hybrid clone detection technique, but there are numerous other popular clone detection tools and techniques. Some of which include Simian\footnote{\url{http://www.redhillconsulting.com.au/products/simian/}}, CloneDR\footnote{\url{http://www.semdesigns.com/products/clone/}}, MeCC\footnote{\url{http://ropas.snu.ac.kr/mecc/}}, CCCD~\cite{Krutz:2015:EEU:2695664.2695929}, and Simcad~\cite{6613857}. We are confident in our selection of Nicad due to its effectiveness which has been demonstrated in previous research~\cite{Roy:2008:NAD:1437898.1438600}. 




%% Why developers create code clones

%%% What has Nicad been used for in the past


%%% Code Reuse
Although code clones have been demonstrated to be detrimental in certain situations, code reuse is imperative for most software development projects. Numerous previous works have studied software reuse on both open source, and commercial applications. Code reuse has been found to save significant time and resources for most projects, along with increasing the overall quality of the software~\cite{493415}. Heinemann~\cite{Heinemann:2011:ENS:2022115.2022138} performed an empirical study in 20 open source projects and analyzed 3.3 MLOC. Their analysis found that 9 of the 20 examined applications had software reuse rates of over 50\%. Fortunately, most of the reuse was through black-box reuse, and not through simply copying \& pasting source code from the various applications. Mockus~\cite{mockus2007large} conducted a study to determine the extent of software reuse in open source projects, identify the most reused code, and investigate patterns of large-scale reuse. This work found that 50\% of the files were being used in more than one project, and that the most widely reused components were generally small, although some were comprised of hundreds of files.

Although there have been a few source code recommender systems~\cite{DBLP:conf/icse/McMillanHPCM12,6340250}, the primary focus of these works are on generic source code, and not tactical code snippets. Therefore the challenges of obtaining and recommending architecturally significant code is still unexplored. This paper conducted a qualitative study and reported challenges related to implementation and reuse of tactical code snippets.


 



%%%%%%%% 
% Make sure to point out what is new
% 	Really point out what our take home message is -- Make it an emph ?
% 






%%%%

%Nicad\cite{Roy:2008:NAD:1437898.1438600}: A text-based hybrid tool that combines the advantages of text and tree-based structural analysis. Clone identification and normalization are conducted using pretty printing and longest common subsequences. Nicad is compatible with C, Java, C#, and Python and has been found to have the ability to detect type-1, type- 2, and type-3 clones.