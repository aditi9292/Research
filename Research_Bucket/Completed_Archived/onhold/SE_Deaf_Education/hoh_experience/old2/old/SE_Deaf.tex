\newcommand{\todo}[1]{\colorbox{yellow}{\textbf{[#1]}}}
\newcommand{\jayme}[1]{\textcolor{red}{{\it [Jayme says: #1]}}}
\newcommand{\dan}[1]{\textcolor{blue}{{\it [Dan says: #1]}}}
\newcommand{\steph}[1]{\textcolor{green}{{\it [Steph says: #1]}}}

\documentclass{acm_proc_article-sp}

\usepackage{cite}
\usepackage{graphicx}
\usepackage{hyperref}
\usepackage{listings}
\usepackage{times}
\usepackage{xspace}
\usepackage{booktabs}
\usepackage{subfigure}
\usepackage{fancybox}
\usepackage{color}
\usepackage{multirow}
\usepackage{array}
\usepackage{subfigure}
\usepackage{balance}
\usepackage{tabularx}

\usepackage{listings}


\begin{document}


% Clean up title
\title{Teaching Deaf and Hard of Hearing Students Software Engineering}


% Clean up authors
\numberofauthors{2} %  in this sample file, there are a *total*
% of EIGHT authors. SIX appear on the 'first-page' (for formatting
% reasons) and the remaining two appear in the \additionalauthors section.
%
\author{
%
% 1st. author
\alignauthor
Daniel E. Krutz\\ 	
	\affaddr{Software Engineering Department}\\
       \affaddr{Rochester Institute of Technology}\\
       \affaddr{1 Lomb Memorial Drive}\\
       \affaddr{Rochester, NY 14623} \\
       \email{dxkvse@rit.edu}
% 2nd. author
\alignauthor
Daniel E. Krutz % salvse\\ 	
	\affaddr{Software Engineering Department}\\
       \affaddr{Rochester Institute of Technology}\\
       \affaddr{1 Lomb Memorial Drive}\\
       \affaddr{Rochester, NY 14623} \\
       \email{dxkvse@rit.edu}
} % Must not be a space above this


\maketitle



\begin{abstract}

GD 


\end{abstract}

\todo{define the categories and fix the keywords \& terms}
% A category with the (minimum) three required fields
%\category{H.4}{Information Systems Applications}{Miscellaneous}
%A category including the fourth, optional field follows...
%\category{D.2.8}{Software Engineering}{Metrics}[complexity measures, performance measures]

\terms{Software Testing, Software Engineering Education, Software Project}

\keywords{ACM proceedings, \LaTeX, text tagging} % NOT required for Proceedings
\section{Introduction}
% What is the problem we are trying to solve.
The Software Engineering Department at the Rochester Institute of Technology (RIT) is the first and largest undergraduate program of its kind in the United States~\cite{FIE_Instill}. RIT is also home to the National Technical Institute for the Deaf(NTID). The goal of this college is to provide technical and professional education to hard of hearing students and is home to over 1500 students~\cite{ntidurl}. Even though NTID is a separate college from the rest of the university, its students often attend classes with hearing students. Interpreters and other necessary resources are made available to all students in an on demand basis. They are available for classroom lectures and activities, team project time outside of class, or whenever else they are requested by the student. NTID is only a two year college, so many of its students will transition away from this college to one of the many other colleges at RIT. Additionally, not all deaf students at RIT belong to NTID.

In this paper, we define a hearing impaired student as a student who meets the legal definition as someone who is hearing impaired and may have partial to full hearing loss. Many of these students have cochlear implants or hearing aides while others do not use any type of hearing assistance devices. 

% Describe the SE department and how things are run
Software engineering is very much team and communication discipline~\cite{Pieterse:2006:SET:1216262.1216282}. At the Software Engineering department at RIT, our courses are typically comprised of a significant team project component. These teams may be comprised of both hearing and hearing impaired students. Other than the additional resources such as interpreters provided by the university, hearing impaired students are treated just like any other students in the program.



% Collaboration both in and outside of the classroom may be hindered due to language barriers. ~\cite{Lang01102002}


\todo{citations}
While we have achieved a considerable amount of success educating hearing impaired students in our software engineering program, we have also faced significant hurdles as well. Despite the best efforts of the interpretors, hearing impaired students typically lose x\% of communication through this avenue. Additionally, hearing impaired students are less likely to speak up in classes. \todo{other issues faced - look at research}. 

%Automatic speech recognition software are accurate at a rate of 75-85\%~\cite{Kheir:2007:IDS:1269900.1268860}


Universities across the United States face significant challenges in educating students with disabilities in computing fields~\cite{Cavender:2009:SAA:1508865.1509043}. Students with disabilities are much less likely to pursue careers in computer science and engineering. Additionally, the dropout rate for these students is high~\cite{4418169}~\cite{national2000women}~\cite{Bueno:2007:ALA:1268784.1268903}. 


% RIT has many resources to assist the hearing impaired and still encounter issues. 


In the following paper, we discuss some of our experiences and future work in creating a robust education experience as possible for both our hard of hearing and non hard of hearing students for the software engineering educational process. We propose the following work in the hope that it may assist instructors at other institutions.


The remainder of this paper is organized as follows. Section~\ref{sec: ourexperiences} describes our experiences and lessons learning with instructing hearing impaired students. Section~\ref{sec: improvments} discusses several improvements which may be made to our program. Section~\ref{sec: relatedwork} provides a list of related works. Section~\ref{sec: futurework} discusses future research to be conducting and section~\ref{sec: conclusion} summarizes the findings of this work.


% Teams are important in software engineering education
% \cite{5319505}
% \cite{1158712}

% How do teams typically communicate in SE



% How to improve deaf education
% \cite{6297673}




\section{Our Experiences}
\label{sec: ourexperiences}
% Describe many of the things that us as software engineering insructors have seen. Also maybe even reference some external computing works as well and compare our findings to what they have found.


%\begin{quotation}
%"Quote 1 "
%\end{quotation}

We asked several students to participate in a survey to help us understand ways to improve the learning environment for not only hearing impaired students, but to acclimate hearing students to better interact with hearing impaired students. Several hearing impaired students were asked to fill out a questionnaire regarding their experiences with working on project teams with not only other hearing impaired students, but predominately with hearing students. Several hearing students who worked on at least one term long project team with at least one hearing impaired team member were also asked several questions regarding their experiences. 








\section{Improvements}
\label{sec: improvments}

Improvements

\section{Related Work}
\label{sec: relatedwork}

This paper represents the first known work on hearing impaired students and software engineering education. However, there are numerous previous papers that discuss hearing impaired education in computing. Ross~\cite{Ross:1982:TPD:964167.964174}  described various methods of teaching programming to deaf students. One of which was through the use of a dynamic library of programming language examples. Several problematic areas for deaf students were also conveyed. Some of which included difficulties with professional note takers and interpreters due to their lack of a computer science background, thus resulting in a significant loss of information.

Cavender~\emph{et al.}\cite{Cavender:2009:SAA:1508865.1509043} described a 9-week summer program for hard of hearing students. This program is designed to provide a catalyst for the academic careers for hearing impaired students. This is largely accomplished through the use of tutors and mentors for these students. Some of the lessoned included the need to inform instructors on how to more properly educate hearing impaired students and the need to recruit tutors and mentors who are themselves disabled so they may better relate to the students. One surprising finding is the communication variations which exist in the deaf community along with a diversity of accommodation needs. Not all hearing impaired students possess the same sign language communication skills, nor do they necessarily have the same preference in sign languages. Students may communicate using American Sign Language (ASL), Signed Exact English (SEE) or "Simultaneous Communication" (Simcomm).

Burgstahler~\emph{et al.}~\cite{4418169} discussed several ways of increasing the participation of students with various disabilities in computing fields. Some of which included the collaboration and knowledge sharing of numerous disability services programs across the United States where strategies for recruiting and retaining disabled students in computing fields is discussed. This work also stated that disabled students were underrepresented in computing and that increasing the participation of disabled students would require a collaborative effort from students, educators and employers.

In order to assist educating hearing impaired students in computing disciplines, several papers have been written. Kheir and Way~\cite{Kheir:2007:IDS:1269900.1268860} discussed using real-time speech transcription in order to assist the inclusion of deaf students in computer science courses. An affordable solution was described which greatly assisted hard of hearing students in these computer science courses. Li and Xu~\cite{5454732} studied an inquiry-based teaching model for deaf students. Inquiry-Based teaching models are very student oriented and allows students to investigate real world computing problems under the direction of the course instructor. This research found that such a model would be beneficial for deaf students.

Bueno~\emph{et al.}~\cite{Bueno:2007:ALA:1268784.1268903} described several methods of assisting instructors in adapting e-learning content. The primary contribution of this work was a tool which processes lecture text for hearing impaired students. This tool works by highlighting words or expressions which are difficult to understand for hearing impaired students and links them to external visual resources. A visual resource is used because numerous studies have found that hearing impaired students who predominately communicate via sign language process images more efficiently than words~\cite{Santos}. This paper also discussed some of the manners in which hearing impaired students learned different compared to hearing students. One is that hearing impaired students learn at their own pace and is very distinct from the pace of their hearing classmates~\cite{Bueno:2007:ECA:1268784.1268862}. 

\dan{Add to this with some of Jayme's work}

\section{Future Work}
\label{sec: futurework}
Future Work


\section{Conclusion}
\label{sec: conclusion}

GD




%\end{document}  % This is where a 'short' article might terminate

\bibliographystyle{abbrv}
\bibliography{SE_Deaf_Refs}  


% That's all folks!
\end{document}



% Site information: http://sigcse2014.sigcse.org/authors/papers.php
% Due Date: 9/6
% Length: 6 pages



% Gather input from students to create 
% Collect more data
% Try out different team education roles
% What team roles do deaf students typically choose
% What roles do they typically do better in?
% Will have the assistance of a speech language pathologist and NTID staff to assist with course planning.



% Data to get
% - What percentage of SE students are deaf? What percentage of students taking SE classes are deaf?
% - How do deaf students perform gradewise
% - 


% Jayme
% - How do deaf students learn better?
% - Proper group interaction strategies with non-deaf students
% - How are these students affected once they reach the "real world". IE how do we get them the most ready?
% - How often do students use these resources?


% Todo
% - Proofread all and make sure it all PC correct
% - capitalize software engineering everywhere
% - What does the poster creation entail
% - Create a website with information on it?
% - 