%!TEX root = ../AndroidData_IEEE.tex
\section{Related Work}
\label{sec:relatedwork}


%%%% Section will need updates based on the venue being submitted to, its length & just to be generally updated
\todo{update section}

% Previous research with permissions
There has been a substantial amount of previous research analyzing the effects of permissions on the user's perception of the app. Felt et al.\cite{Felt:2012:APU:2335356.2335360} performed two usability studies from users over the internet and in a laboratory setting and found that only 17\% of users paid attention to permissions when installing apps. Lin et al.\cite{Lin:2012:EPU:2370216.2370290} conducted a study analyzing user's comfort levels when apps requested permissions which the user's did not understand why the apps needed them. They found that user's generally felt uncomfortable and may even delete applications when they did not understand why it requested a permission they deemed unnecessary. Egelman et al.\cite{Egelman12choicearchitecture} found that approximately 25\% of users were typically willing to pay a premium in order to use the same application, but with fewer permissions, while about 80\% of users would be willing allow their apps more permissions to receive targeted advertisements if it would save them .99 cents on the purchase of the app. Stevens et al.~\cite{Stevens2013} found that certain permissions that are not popular among developers were frequently misused, possibly due to a lack of documentation. Similar to the above papers, we examine the permissions and security risks in Android apps. However, we compare them against user ratings, which are user's perspective of an app that was obtained without us soliciting it from them. 


App ratings have demonstrated their importance in other areas of research as well. Harman et al.\cite{6224306} found a strong correlation between the rating and the number of app downloads. Linares-Vasquez et al.\cite{Linares-Vasquez:2013:ACF:2491411.2491428} found that change and fault-proneness of the APIs used by the apps negatively impacts their user ratings. Khalid et al.\cite{Khalid_Mei_Examinging} examined 10,000 apps using FindBugs and found that warnings such as~\lq Bad Practice\rq, ~\lq Internationalization\rq, and~\lq Performance\rq categories are typically found in lower rated apps. Their primary discovery was that app developers could use static analysis tools, such as FindBugs, to repair issues before users complained about these problems. Even though we too use ratings as an evaluation measure, we look at permission and security risks unlike earlier works. 
%Kim et al.~\cite{ouca13_2}  found ratings to be a determining factor in a user's decision whether or not to purchase apps. 
%\todo{Mei: Check this statement - was a bit confused on what you were saying here.}


%\todo{make sure to add : To Appear }

%\dan{Mei: Obviously feel free to alter this}



% Previous was a feasiliblity study




% Cite Mei's work?


% ? Any studies that look at the rate of overprivs and the effects on the user?


% Much of these findings will be dependent upon where we go with the work




% Talk about the problems of overpermissions
