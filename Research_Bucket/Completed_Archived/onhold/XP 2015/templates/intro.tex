%Agile Architecture, no upfront design

%Design thinking integrated into coding

%Advocates design thinking while programming

%No documentation of architecture

%Design Knowledge is tacit in the head of people

%Use social ways to maintain knowledge



%% Introduction to Agile development methodologies

Agile software development is a method of managing software projects to make them more adaptable to change and to reduce costs associated with unnecessary documentation. Agile development emphasizes close collaboration between developers and business experts, frequent face to face communication, continual delivery of incremental working software, and self organizing teams. This is a stark contrast to development techniques such as Waterfall, that typically require significant amounts of up front requirements analysis, architecture design and continuous documentation. This often makes project changes more difficult and expensive.

The reduction of excessive up front design and unneeded formal architectural documentation increases the necessity for other methods of addressing quality, communication, and interactions between developers and stakeholders. In agile software development, architecture design is integrated with coding activities, advocating less up front design and more incremental architectural spikes to address each quality requirements. In such an environment, there is also less emphasize on documenting architectural knowledge from driving requirements to the adopted patterns, tactics and styles. The rationale behind these choices are maintained socially through practices such as pair programming, collective ownership and stand up meetings.

Currently, agile techniques are supported with tools which promote core agile practices. For example there are several mature tools for code refactoring and test driven design~\cite{Alves:2014:RRA:2635868.2661674, Moghadam:2011:CTA:1984732.1984742, Nongpong:2012:ICS:2519037}. Unfortunately, there are few tools which assist developers in agile architecture development, design maintenance~\cite{ICSE2012,Erosion} and documentation. Architecture centric tools~\cite{KA,AntonyTool,DBLP:conf/qosa/LeeK08} assume that an up front architecture design processes exist and heavy design artifacts are created. This makes them impractical in many iterative incremental projects with small cycles of design and often leads developers to not adopt these tools. Instead, they are often forced to rely upon social techniques to develop, communicate, and maintain their architecture.

The lack of architecture centric tools that fit the agile development paradigm and culture can increase the chances of quality degradation, where the implementation of architectural choices to satisfy quality concerns are drifted from the initial intends resulting design erosion and degradation of software qualities. In Robert Martin's recent book entitled ``Agile Software Development - Principles, Patterns, and Practices''~\cite{Martin2002}, he commented that while developers may initially release a system that meets the intended design, it does not take long before ``the software starts to rot like a piece of bad meat'' leading  to problems such as rigidity, fragility, and unnecessary complexity of the design. 

This problem is exacerbated by the fact that popular software engineering tools and environments fail to make underlying design decisions visible to programmers. This results in maintainers not being kept fully informed of the relevant underlying patterns, tactics, and constraints as they build, maintain, and refactor a software system~\cite{Booch:DrawPicture}.

In this paper, we introduce a pluggable tool~\emph{Archie}, which can be used to support different agile-architecture development activities. This tool was initially developed to support integrated architecture development and maintenance. However the automation features of Archie makes it suitable for agile projects. Archie has features for helping developers devise incremental architectural choices, proactively sharing design knowledge with programmers, and keeping them informed of underlying architectural decisions during coding activities. Archie helps developers to perform change impact analysis of architectural concerns at the code level, and provides infrastructure to enable the concept of ``Design Ownership'', supporting the developers to obtain accountability for their design choices.

