
Several other works have discussed the importance of design in agile software development. Wirfs-Brock~\cite{Wirfs-Brock:2010:SAD:1869542.1869630} spoke about the significance of agile developers having the ability to quickly understand design problems and make timely decisions about how to address the issue. 

While our work is innovative, there are a wide range of existing techniques that support agile software development. Spoelstra et al.\cite{Spoelstra:2011:SRA:1982185.1982255} proposed a conceptual management tool for supporting software reuse in agile software development using a components derived from the Software Process Improvement (SPI) framework by Niazi et al.\cite{Niazi:2005:MMI:1045926.1045932}. Their tool utilizes maturity levels and reuse practices in order to create a reuse maturity model. However, none were specifically focused on agile architecture development. There is a wide range of other tools supporting agile software development including tools for software project management and planning~\cite{Petersen:2008:APT:1379092.1379101, Dhlamini:2009:IRM:1562741.1562745}, and modeling~\cite{Buchmann:2012:TTS:2467307.2467310}. 

Several researchers have attempted to address architectural challenges through developing techniques for organizing, documenting or modeling architectural decisions. The Architecture Design Decision Support System (ADDSS)\cite{ADDSS}, Process based Architecture Knowledge Management Environment (PAKME)\cite{PAKME}, and Architecture Rationale and Element Linkage (AREL)\cite{AREL} are examples of these. However, these approaches are not adopted to the agile development mindset, and fail to address the scalability issues of managing potentially large numbers of architectural decisions.  They also fail to connect design decisions to code, and/or provide little support for actually utilizing this knowledge during software maintenance. 

