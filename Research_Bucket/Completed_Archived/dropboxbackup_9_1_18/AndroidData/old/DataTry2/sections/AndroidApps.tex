

The Android operating system is the most popular mobile platform in the world with apps being available on numerous types of devices from a variety of manufacturers~\cite{androidpopularity_url}. This flexibly has allowed the Android operating system to flourish, but results in many different hardware platforms and OS versions for app developers to support.

\subsection{Android Application Structure}

The Android application stack is comprised of four primary layers. The top layer is the Android application layer, which is followed by the the three application framework layers. The Android Software Development Kit (SDK) allows developers to create Android applications using the Java programming language. Isolation between Android applications is enforced through the use of the Android sandbox~\cite{androidsecuritytips_url}, which typically prevents applications from intruding upon one another.

\emph{Intents} are a communication mechanism to exchange information between the components (\emph{Activities}) of an Android application, and are reported to the user upon installation and use. Inter Process Communication (IPC) is the composition mechanism performed using Intents which is used to invoke another application component. Attacks that exploit Intents for malicious reasons include ~\emph{permission collusion},~\emph{confused deputy}, and~\emph{intent spoofing}~\cite{Chin:2011:AIC:1999995.2000018,grace2012systematic, Marforio:2012:ACC:2420950.2420958, 6641043}.

Android applications are packaged in APK files, which are compressed application files which includes the application's binaries and package metadata. Table~\ref{Table:apkcontents} shows the breakdown of a typical APK file.

%% Much of this able came from : ~\cite{Lee_2013}


\begin{table}[ht]% Try here, and then top
\begin{center}
\caption{APK Contents}
\label{Table:apkcontents}
  \begin{tabular}{| l | l | } \hline

    \bfseries File & \bfseries Description \\ \hline
    AndroidManifest.xml & Permissions \& app information \\ \hline
    Classes.dex & Binary Execution File \\ \hline
    /res & Directory of resource files \\ \hline
    /lib & Directory of compiled code \\ \hline
    /META-INF & Application Certification \\ \hline
    resources.arsc & Compiled resource file \\ \hline
  \end{tabular}
  \end{center}
\end{table}


Android applications are available from a variety of different locations including AppksAPK\footnote{http://www.appsapk.com/}, F-Droid\footnote{https://f-droid.org/}, and the GooglePlay store\footnote{https://play.google.com/store}. These stores differ from the iOS app store, which forces all non-jailbroken devices to access applications through an Apple controlled store. GooglePlay provides verification of uploaded applications using a service called Bouncer which scans applications for malware~\cite{bouncer_url1}. In spite of these efforts, malicious apps are sometimes found on the GooglePlay store~\cite{Zhou:2012:DAM:2310656.2310710}. GooglePlay separates apps into~\emph{Genres} based on their realm of functionality, some of which are Action, Business, Entertainment, Productivity, and Tools. The~\emph{AndroidManifest.xml} file contains permissions and application information as defined by the developer.

\subsection{Android Permission Structure}
Android developers operate under a permission-based system where apps must be granted access to various areas of functionality before they may be used. If an app attempts to perform an operation which it does not have permission, a~\emph{SecurityException} is thrown. When an Android app is created, developers must explicitly declare in advance which permissions the application will require~\cite{Felt:2011:APD:2046707.2046779}, such as the ability to write to the calendar, send SMS messages, or access the GPS.

When installing the application, the user is asked to accept or reject these requested permissions. Once installed, the developer cannot remotely modify the permissions without releasing a new version of the application for installation~\cite{shaerpour2013trends}, prompting the user if new permissions are required. These security settings are stored in the AndroidManifest.xml file and include a wide range of permissions, some of which are~\emph{INTERNET},~\emph{READ\_CONTACTS}, and~\emph{WRITE\_SETTINGS}. Unfortunately, developers often request more permissions than they actually need, as there is no built in verification system to ensure that they are only requesting the permissions their application actually uses~\cite{Felt:2011:APD:2046707.2046779}.

A basic principle of software security is the~\emph{principle of least privilege}, or the granting of the minimum number of privileges that an application needs to properly function~\cite{saltzer1975protection}. Granting more privileges than the application needs creates security problems since vulnerabilities in other applications, or malware, could use these extra permissions for malicious reasons. Additionally, this limits potential issues due to non-malicious developer errors. Unfortunately, due to the lack of granularity of the permission spectrum used by Android, the developer must often grant more permissions to their application than it actually requires. For example, an application that needs to send information to one site on the internet will need to be given full permissions to the internet, meaning that it may communicate with with all websites~\cite{jeon2011dr}.

In this study, we use the term \emph{overprivilege} to describe a permission setting that grants more than what a developer needs for the task. Likewise, an \emph{underprivilege} is a setting for which the app could fail because it was not given the proper permissions. Overprivileges are considered security risks, underprivileges are considered quality risks.



