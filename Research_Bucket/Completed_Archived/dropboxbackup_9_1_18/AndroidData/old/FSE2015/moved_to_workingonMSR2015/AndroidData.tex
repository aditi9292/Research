
\documentclass{sig-alternate}
  \pdfpagewidth=8.5truein
  \pdfpageheight=11truein

%\documentclass{sig-alternate-2014}

% Good site for bibstrip information
% http://tex.stackexchange.com/questions/150055/how-to-add-copyright-box-in-acm-alternative-style


\usepackage{cite}
\usepackage{graphicx}
\usepackage{listings}
\usepackage{caption}
\usepackage{times}
\usepackage{color}
\usepackage{multirow}
\usepackage{xcolor}
\usepackage{url}
\usepackage[numbers]{natbib} % Used to fix formatting issue.
\usepackage{soul} % Needed for wrapping of highlighted text
\usepackage{balance} % Used to balance out the columns

%\usepackage{pxfonts}
%\usepackage{xspace}
%\usepackage{booktabs}
%\usepackage{fancybox}
%\usepackage{array}
%\usepackage{tabularx}
%\urlstyle{same}
%\usepackage{pgfplots}
%\usepackage{tikz}
%\usetikzlibrary{shapes,arrows, positioning}
%\usetikzlibrary{patterns}



\definecolor{bblue}{HTML}{4F81BD}
\definecolor{rred}{HTML}{C0504D}
\definecolor{ggreen}{HTML}{9BBB59}
\definecolor{ggrey}{HTML}{707070}


\newcommand{\todo}[1]{\textcolor{cyan}{\textbf{[#1]}}}
\newcommand{\sam}[1]{\textcolor{red}{{\it [Sam says: #1]}}}
\newcommand{\dan}[1]{\textcolor{blue}{{\it [Dan says: #1]}}}

\long\def\/*#1*/{}   % Used for comments
%% Start: \/* This is a test
%% End: */


\begin{document}

% Maturity and Security: Static Analysis of Reverse-Engineered Open Source Android Applications

\title{XXXXXXXX}
\numberofauthors{1} %  in this sample file, there are a *total*
% of EIGHT authors. SIX appear on the 'first-page' (for formatting
% reasons) and the remaining two appear in the \additionalauthors section.
%
\author{
% You can go ahead and credit any number of authors here,
% e.g. one 'row of three' or two rows (consisting of one row of three
% and a second row of one, two or three).
%
% The command \alignauthor (no curly braces needed) should
% precede each author name, affiliation/snail-mail address and
% e-mail address. Additionally, tag each line of
% affiliation/address with \affaddr, and tag the
% e-mail address with \email.
%
% 1st. author
\alignauthor
Daniel~E.~Krutz and xxxxxxxxx\\ 	
	\affaddr{Software Engineering Department}\\
       \affaddr{Rochester Institute of Technology}\\
     %  \affaddr{1 Lomb Memorial Drive}\\
     %  \affaddr{Rochester, NY 14623} \\
       \email{\{dxkvse, XXXXX\}@rit.edu}
 }

\maketitle
\begin{abstract}

Mobile devices have not only changed the way we use computing, but also the way we live. Android has grown to be the most popular platform in the world, largely to its flexibly to work on a wide range of devices and the ability for user's to install applications (apps) from a wide range or sources. 

The mobile revolution has opened the door for a variety of new types of apps and new developers into the apps race. Unfortunately, mobile apps are not immune to the problems which plague conventional software including bugs, and security vulnerabilities. Examining version control systems (VCS) of open source applications is a good way of understanding when, why and who introduced defects and various types of security vulnerabilities.

In the following work, we examine over \todo{XXXX} open source applications and over \todo{XXXX} versions of these applications in order to gain a better understanding of why bugs and security vulnerabilities are created in apps, when they typically appear in the lifecycle of the apps, and if the same vulnerabilities typically reappear in apps.


% Do we just want to focus on security & over permissions and not bugs?
% Mention about how we look at developers and commit histories?

\end{abstract}


% I think this is the most appropriate
%\ccsdesc[500]{Security and privacy~Software and application security}
\category{D.2.7}{Software Engineering}Maintenance;
%They are here: http://www.acm.org/about/class/ccs98-html


%%% This was left out by 2/3 printed examples, so it may not be a bad idea to leave it out as well
%%%     Also saves space.
%\terms{xxx, xxx, xxx}


\keywords{Code Clones, Concolic Analysis, Software Engineering}



% Add in categories and keywords


\section{Introduction}








\section{Related Works}
\label{sec: relatedworks}


% Works that look at commit messages
%	GIT histories
% 






\section{Research Questions}
\label{sec: researchquestions}


RQ1: How does time affect security and quality of the app?
% Average commit times - This could be just where the timezone is
%	In order to do this, we will need the timezone information



RQ2: How do committers affect the quality of an app?
- Diversity (number) of developers
- Experience of developers
- Are some developers more
- Work across many applications 





RQ3: What tendencies do Overprivs have in apps?

- Exist at beginning and not get fixed.
- Exist at beginning, but are fixed.
- Not exist at beginning and are later added to app
- Are fixed and then come back
- If there is one, are there many?
%	- This is answered below


%%% Look at all the versions for the apps to see how many were overpriced at one time
% 	At least 4 versions
% See: AppsWithXOverPRivs.csv
%	This looks at all the versions of the apps
% #### QUERY 6
% 339			0			217			122			122/339=36% had at least 1 over priv in their versions
%Total		AllOverPriv	NoneOverPriv	SomeOverPriv




%
%- Exist at beginning and not get fixed.
% OverPriv in 1st version, an overdrive in all versions










% #### QUERY 3
- \% of apps with at least 1 over prove in development cycle - See "XOverPrivCount.xls"
- 122/339 = 36\%
- Latest version of app



% Compare this with the findings of Felt et al.
%	This count is for the latest version
49/339 had at least 1 over priv in their final version
-	Avg with at least 1 = 1.84
-	
Count with at least	1	49
				2	18
				3	8
				4	5
				5	4
				6	2
				7	2
				8	2
				9	0



%% Commit messages vs. Over Priv rate
%% ? What do other papers that talk about comments discuss? - Follow their lead for this work
% 













% #### QUERY 4
%-- If an App has an overpriv, how likely is it to be underprived?
%%% 
% 315 versions of apps had 1 overpriv. 270 of these were underpriced. This happened in 
% 

		
		
		

% What are the most over used over privy
%	Are these mentioned in comments at all










%%% Correlation of # of committers vs. size of the app and its relation to being over prived?











%%% ^^^^
% Get numbers and instances for each
%	Of all apps and just those that are over prived
% The top are basic questions to ask, while the bottom is how they can be elaborated on and analyzed.





% Do apps become more buggy over time
% Does the over permission rate in apps grow over time
% Does the vulnerability rate grow over time?
% Commit messages about bugs vs. vulnerabilities?
% Are apps with at least 1 overpriv likely to have more? - Where there is one, there are many?




% ? Can i take any RQs from older work that are still applicable?


\label{sec: androidapplications}
\section{Android Applications}

% Use much of the same data from the ICSE paper



\label{sec: collection}
\section{App Collection \& Static Analysis}






\section{Publicly Available Dataset}
\label{sec:dataset}

% Keep this section reasonably short



\section{Evaluation \& Analysis}
\label{sec: evaluation}



%RQ: Do over permissions get fixed and not come back?
% 	How many instances does this happen in
%	What are the over permissions
%	Analyze the VCSs and why these happen


%RQ: Are Overprivs usually fixed, or do they get created and persist?
%	Give a few intances
%	Talk about general trend
%	Give some numbers 
%		In XXX possibilities, we observed XXX cases were oprivs were not fixed
%	Dig into why this happens
%		Look at commit messages
%	? Is there a general trend of which are over prived?




% Do certain people always worry about security - in same apps and accross many apps?
% Find instances where security problems were fixed
%	Compare to why they went bad?


% Do Time of Day and Developer Experience Affect Commit Bugginess?
%	Has good definitions for commit information I can follow
%		Commit info
% 			What order are items done in? Bug before vulneraiblity?
%			How do messages correlate with how the actual app is doing?
%
% Do the same changes/comments happen accross different apps from different committers?
%		Do they happen with the same committers? IE, do the same committers make the same mistakes over and over again?
%			- This is probably very tough to tell 
% Is there a way we can tell that committers use different names
%	Same messages, same times, different author names?


% Specific instances for case study - at least 4 versions of app
% 	Apps where permissions got fixed and then problem happened again
%	Apps were an overpriv was added
%		9,22, 23, 24, 51, 76, 90, 131, 235, 277, 478, 496, 556, 597, 612, 653, 681, 724, 1137
%			This is an incomplete list of numbers
%		9:  0,0 -> 1,7 -- Analyze why this happened
%		18: 0,0->1,4->2,2->2,0
%		23: 0,0->3,4
%		51: 0,0 ->2,2
%		56: 0,0 ->1,8
%		76: 0,0 ->2,4
%		90: 0,0-> 1,6
%		98: 0,0-> 4,2
%		111: 0,0->1,2
%		130: 0,0-> 1,6
% ----- Stopped here --- See query below
%		
%	Thoughts:
%		Make sure app is not too small with too few committers


%	App was fine, overpriv added, overpriv fixed (Run queries)
%		15
%	Overpriv always existed, never fixed
%	Overpriv existed at start, and then got fixed
% 	Priv was bad, and then was at least partially fixed
%		68 


%%% Temp query = 1/2


%%%************************************************************************
\/* 


select ad.appID
, case when overprivCount.oprivcount is null then 0 else overprivCount.oprivcount end as OPriv
, case when underprivCount.uprivcount is null then 0 else underprivCount.uprivcount end as UPriv
--, case when vul.fuzzy_risk is null then 0 else vul.fuzzy_risk end as FuzzyRisk


from appdata ad

inner join version v on v.appID = ad.appID

left outer join (select count (VersionID) as VersionCount, appID from Version v2 group by appID) VersionCount on (VersionCount.appid) = ad.appID

left outer join (select count (permissionID) as oprivcount, versionID from overpermission group by versionID) overprivCount on (overprivCount.versionID) = v.versionID

left outer join (select count (permissionID) as uprivcount, versionID from underpermission group by versionID) underprivCount on (underprivCount.versionID) = v.versionID

left outer join Vulnerability vul on v.versionID = vul.versionID

left outer join
(
	SELECT appID, t.build_number, versionID, (SELECT COUNT(*) FROM version counter
           WHERE t.appID= counter.appID AND t.build_number >= counter.build_number) AS row_num
	FROM version t
	order by appID, row_num

) vi on vi.versionID = v.versionID
where VersionCount.VersionCount >= 4
--and row_num <= 5
order by ad.appID, row_num

%%%************************************************************************





% Notes:
%	Apps had at least 5 versions
%	


select ad.AppID, v. VersionID, ad.Name, p.name
from appdata ad
inner join version v on v.appID = ad.appID
inner join overpermission op on op.versionID = v.versionID
inner join permission p on p.PermissionID = op.PermissionID
where ad.appID = 18
order by v.versionID
*/
%%%%%%%%%%%%%%%%%%%%%%%%%%%%%%%%%%%%%%%%%%%%%%%%%%%%%%%%%%%%%

\/* This is a test
-- See Google DOC

*/




%%% Break down to the apps into stat groups depending on the versions
\/* 

SELECT *, (SELECT COUNT(*) FROM version counter 
           WHERE t.appID= counter.appID AND t.versionID >= counter.versionID) AS row_num
FROM version t
order by appID, versionID








*/

\section{Limitations}
\label{sec: limitations}

% Do something to talk about how the recompilation process is robust & works without too many difficulties
%	Much of the feedback from the ICSE paper 

\todo{update all of this from ICSE paper}
While Stowaway is a powerful statical analysis tool which has been used in a substantial amount of previous research~\cite{Pearce:2012:APS:2414456.2414498,Stevens_investigatinguser,jeon2011dr}, it does suffer some drawbacks. Malicious code may be obfuscated and unnecessary API methods inserted into the application, rationalizing the permission~\cite{6698893}. Static analysis techniques can also be hindered by the Java reflection and may lead to inaccuracies~\cite{Sridharan:2006:RCP:1133255.1134027,Tripp:2009:TET:1542476.1542486}. These types of limitations are inherent to all statical analysis tools.

We only analyzed applications from GooglePlay and not other sources such as AppksAPK or F-Droid, which would have led to more varied application origins. However, we feel the diversity of our applications was already quite robust since we collected 30,020 applications from 41 genres.

We also only examined free applications in our research due to cost constants. Thus, the measurements comparison of apps is not representative of the entire Android app market. Our results only apply as a comparison of free apps, not with paid apps.






% Make it one section
\section{Future Work}
\label{sec: futurework}

% Analyze more apps and use more tools
% Look at the actual code that is changed
%	Was not done here, an immense amount of information to collect
% 


\section{Conclusion}
\label{sec: conclusion}





\balance
\bibliographystyle{abbrv}
\bibliography{AndroidData} 

\balancecolumns
% That's all folks!
\end{document}

%%%% Notes
% 


%%%% Todo
% 	More analysis on what is going on
%	Actually look into the repos 
%	Update categories and keywords
%	Commit messages & developer histories?