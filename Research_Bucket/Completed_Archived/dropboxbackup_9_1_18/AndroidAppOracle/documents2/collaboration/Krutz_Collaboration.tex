\documentclass[11pt]{article}
%\usepackage{fullpage}
\usepackage{times}
%\usepackage{url}
\usepackage{color}
%\newcommand {\doublespace} {\addtolength{\baselineskip}{.5\baselineskip}}
%\newcommand {\singlespace} {\addtolength{\baselineskip}{-.333\baselineskip}}



\usepackage{cite}
\usepackage{url}
\usepackage{xcolor}
\usepackage{pgfplots}

\usetikzlibrary{patterns} %% Used for bar charts
\newcommand{\todo}[1]{\textcolor{cyan}{\textbf{[#1]}}}


% Define flow chart styles
\tikzstyle{decision} = [diamond, draw, fill=blue!20,
    text width=15em, text badly centered, node distance=3cm, inner sep=0pt]
\tikzstyle{block} = [rectangle, draw, fill=blue!20,
    text width=15em, text centered, rounded corners, minimum height=4em]
\tikzstyle{line} = [draw, -latex']

\usetikzlibrary{shapes,arrows, positioning} % Needed for analysis diagram


\begin{document}
%\pagestyle{empty}
%\renewcommand{\thepage}{DMP-\arabic{page}}
\setcounter{page}{1}

\def\myskip{3ex}

%\centerline{\normalsize Rochester Institute of Technology}
%\vspace{4 mm}
\centerline{\Large\bf PLASMA: Practical LAbs in Security for Mobile Applications}


%% Should be 2 pages in length

\section{Roles of the PIs}

PI Krutz and Co-PI Dean have known each other for approximately ten years, and frequently discuss pedagogical topics. PI Krutz and Senior Personnel Mirakhorli have worked together in the Department of Software Engineering at RIT for over three years and have collaborated on two publications~\cite{SAC2016, MSRDan}. Consultant Richards has been a regular guest speaker in PI Krutz's courses for the last four years. PI Krutz and Senior Personnel Sparkman regularly meet to plan, and evaluate some of the preliminary outreach events and assessment information.


%PIs Kazman, Cai, Mirakhori, and Ryoo already have a long and successful history of collaborating, both in general and on NSF-supported research (Awards 1140300 and 1065242).  Our collaboration over the past five years has resulted in two journal papers, three conference papers, one book chapter, a tool demo paper, one workshop paper, and one pending patent. In addition, three other papers are currently under submission.

PI Krutz will be responsible for leading the initial module creation efforts at RIT. Senior Personnel Mirakhori and Consultant Richards will provide guidance on which modules should be created and provide feedback on the created modules to PI Krutz. After the modules have been evaluated, PI Krutz and Senior Personnel Mirakhori will work with Co-PI Dean to integrate these modules into the curriculum at RIT and Park University. Senior Personnel Mirakhori will provide guidance to Co-PI Dean during Park University's Cybersecurity curriculum development. Senior Personnel Sparkman will work with other project personnel to collect and share educational findings and ensure that the project is meeting the defined educational objectives.

% These interactions will also create collaboration between RIT's Center of Cybersecurity and Park University.


%% Make sure to mention how the guidance will be provided ?
%For the research being proposed here we expect that our roles and relationships will remain largely as follows.  PIs Cai and Mirakhori with the help of their graduate students will lead the design and development of the detection techniques and their underlying theoretical foundations, as well as empirical analysis of the detection technique's primary outputs. PIs Kazman and Ryoo, with the help of their graduate students, will lead the development of the tactics realization ontology and application AAFS in real-world architectural analyses, focusing on empirical studies of the manual and automated aspects of refactoring architectural hotspots for security.
%
%However this division is simply a matter of leadership: all PIs expect to continue their history of close collaboration on {\em all} research questions, and all expect to co-supervise or be on the PhD committee of one another's students.


\section{Management Plan}


%Our collaboration has already been successful due to a number of factors: we have a practice of engaging in frequent teleconferences and web-conferences (the entire research group meets once a week, and PIs Kazman, Cai, Mirakhori, and Ryoo speak and email regularly in between those weekly meetings), frequent group status update emails, and regular face-to-face meetings (at conferences and workshops and at our respective locations).  In addition we maintain shared repositories and shared tools.  Kazman is also currently co-supervising three of Cai's PhD students.   We have no reason to expect that any of this will change in our collaborations for the proposed research.
%

Our collaboration has already been successful due to a number of factors: PI Krutz and Senior Personnel Mirakhori meet in person several times a week to discuss project matters. PI Krutz and Co-PI Dean interact several times a week via emails and phone calls to discuss the project, and how it will be used in collaboration between RIT and Park University. We intend to continue our tradition of regular group web-conferences and in person meetings involving all project personnel and participating students. Interactions will also take place at venues where we are disseminating our work in the form of publications, and through tutorials at various conferences. Furthermore, we will continue our use of cloud-based repositories that allow us to easily share data, papers, and other project results.


%We have no reason to expect that any of this will change in our collaborations for the proposed research.

%The project management will build on our current successes and models.  We intend to continue our tradition of weekly group web-conferences involving all PIs and all students, and quarterly face-to-face meetings, typically lasting 2-3 days. Not all of these meetings are at our home universities; we expect to continue our successful practice of meeting, and working, at national and international conferences such as ICSE, WICSA, FSE, and other appropriate venues as these save on time (since we are already attending those conferences) and save on travel funds.  Furthermore, we will continue our use of cloud-based repositories that allow us to easily share data, papers, and other research results.



%
%To support these collaborations and mechanisms we have budgeted regular conference trips, both national and international, and travel to one another's locations.  We have also budgeted a small amount of funds for computer purchases and software services (to pay for the shared cloud-based data repositories and for the occasional purchase of software tools to support our research).


% To support these collaborations and mechanisms, \todo{Ideas on what I could say?}
%	


%
%Finally, our research is, in the end, empirically grounded.  We can not confidently validate our results unless we work closely with industrial partners.  And so we expect that we will make occasional trips to work together, with industrial partners, at their locations.  We already have built substantial relationships with industry, and we expect to deepen and broaden this  external collaboration in our proposed research program over the next four years.


%Our assessment information will be gathered from classroom events at RIT and partner institutions, along with at outreach events. Without this information, we cannot confidently validate our educational objectives

%PI Krutz and Senior Personnel Sparkman have already been



%\todo{Mention something about WiC and NC AT ?} % Talk about how we will be working with NC A&T with this.

% It is ok to mention how I will delegate responsability to WIC. Describe how the responsabilities will be delgated.
% 	Do the same for NC AT

Our proposal includes support for outreach events through NC A\&T (HBCU) and Women in Computing (WiC) at RIT. To provide guidance to NC A\&T, we will work with Professor Chris Doss. We will hold regular Skype sessions to train both Professor Doss, and his student employees on the most appropriate ways to conduct the activities and collect appropriate module feedback. Using WiC volunteers, we have already conducted several events for WiC students and underrepresented urban High School students in the Rochester, NY area. We will hire one WiC student to be the primary manager of outreach activities, and two other WiC students to help conduct outreach events. Although we will be in regular communication about each event, selecting a WiC student to manage these activities will allow us to delegate these responsibilities and allow the WiC student to gain valuable project management experience.


\section{Timeline Highlights}
%The projected timeline of our project is shown in Figure~\ref{fig:timeline}. Most of the tasks overlap since the detection of tactics and vulnerabilities, anti-pattern detection and solution, as well as their evaluation have to explored side-by-side.  Our focus in the first two to three years is to implement and extend the key abstractions and algorithms.  In the latter portion of the grant, we will focus more on empirical validation, the creation of data acquisition infrastructure, the creation of sharable empirical results and data sets,  the construction and usability evaluation of AAFS, and technology transfer.



According to the described time line in the project description, we have described a plan for our project, which is dependent on the collaboration as outlined in this document. The modules will be defined, created, evaluated and released in an iterative fashion. This will allow us to receive early feedback on our modules, allowing for their early inclusion both in the classroom and at outreach events. An overview of our module creation process is shown in Figure~\ref{fig:developmentprocess}.

%%% Remove due to space
% Although development will take place throughout the two years of the project, during the first year more emphasis will be placed on module creation while in the second year there will be efforts devoted to advertising and disseminating the project results.


%\todo{create timeleine of events and evolution of module creation?}
%\todo{According to the figure in the project description, we have described a plan for our project. This plan is dependent on the collaboration as outlined in this document.}



%



%\begin{figure*}[ht]%{l}{.5\textwidth}
%\centering
%	\includegraphics[width=1\textwidth]{figures/timeline.png}
%\caption{Project Timeline }
%\label{fig:timeline}
%\end{figure*}


\tikzstyle{line} = [draw, -latex']
\tikzstyle{cloud} = [draw, ellipse,fill=white!20, node distance=3.2cm,
    minimum height=2em]

	\begin{figure}[h]
	\begin{center}

\begin{tikzpicture}[node distance = 10 cm, auto]
    % Place nodes
     \node [cloud] (init) {development};
     \node [cloud, below right of=init] (dex) {implementation};
     \node [cloud, below left of=init] (fb) {feedback};

     \path [line] (init) -- node {release}(dex);
     \path [line] (dex) -- node {provides}(fb);
     \path [line] (fb) -- node {provides}(init);

\end{tikzpicture}
\caption{Development Process}
\label{fig:developmentprocess}
\end{center}
\end{figure}




\bibliographystyle{abbrv}
\bibliography{Collaboration}
\end{document}

