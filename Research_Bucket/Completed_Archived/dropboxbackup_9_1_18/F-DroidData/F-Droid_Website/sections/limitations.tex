% Unable to represent all types of apps
% Every institution and developer are different, cannot account for all situations


While we feel that our dataset is robust and quite useful for a variety of areas of future research, it does have some limitations and areas that will be improved upon. In the future, new static analysis tools may be added to examine the apps. Since we are storing the version control histories and source code locally, running these tools retroactively against previous versions of apps should require little effort.

While we feel that collecting more than 1,000 apps with more than 4,000 versions and over 430,000 total commits represents a substantially sized dataset, there are over 1.4 million~\cite{appBrain_stats} apps available on GooglePlay, so our collection represents only a very minor portion of all apps. Additionally, we only analyzed free apps, excluding paid apps in our dataset.

Future improvements will be made to the website not only making it more usable, but also displaying the data in a more user friendly manner as well as allowing more customizable reports.

While we have a high level of confidence in our chosen static analysis tools, static analysis tools are inherently imperfect and typically suffer from a variety of problems~\cite{chess2004static}. Some of which are that they can never demonstrate the correctness of an application, only that it could not find any problems, the need for human verification, and they often produce false positives and false negatives.


