The success or failure of a mobile application (`app') is largely determined by user ratings. Users expect apps to continually provide new features while maintaining quality, or the ratings drop. However, apps must also be secure. But is there a historical trade-off between security and ratings? Or are app store ratings a more all-encompassing measure of product maturity? We collected and compared several security related metrics from 798 random Android apps in the GooglePlay store with a user rating of less than 3 against 861 apps with a user rating of 3 or greater. From our experiments, we find that several of the security metrics that we collected using static analysis were higher (more risk) in high rated apps compared to low rated apps. For example, while low rated apps request more permissions overall, high rated apps will have more over-permissions and under-permissions than low rated apps. Combining the collected evidence from two static analysis tools, we conclude that, historically, user ratings may be a more all-encompassing measure that does not depend heavily on security (yet). 
%and potential security risks may trade off. \todo{make sure these results match what we found}
%A focus on features can lead to security tests slipping. 
%\dan{short sentence?} At a glance, one may assume that user ratings and security trade off. 
%Among our results, we found that lower rated apps generally had a higher rate of overly open privileges. 

