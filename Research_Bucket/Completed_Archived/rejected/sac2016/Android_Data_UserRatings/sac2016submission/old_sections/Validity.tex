\section{Threats to Validity}
\label{sec:threats}


%%%
\todo{update section}


While GooglePlay is the largest Android market place, it is not the only source for Android apps. Alternatives include the Amazon app store, GetJar, F-Droid and a multitude of other sources. Other studies may choose to include apps from these other sources. Additionally, we chose apps at random and only selected a total of 1,659 apps, which is a small minority of the over 1.5 total Android apps available\footnote{{\url{http://www.appbrain.com/stats/number-of-android-apps}}}. Given that this is a random sample, however, we believe that it is representative of the Android application population.

Although Stowaway and Androrisk have been used in a substantial amount of previous research, they are by no means perfect or are the only over-permission and under-permission, and risk assessment tools available. While no tools are likely perfect, Stowaway has only been found to achieve approximately 85\% code coverage of the Android API~\cite{Felt:2011:APD:2046707.2046779}.

While user ratings have been substantially analyzed in previous research, popular apps will receive ratings from only 2-3\% of its users~\cite{Yan:2011:APM:1999995.2000007}. This means that only a small subset of users are responsible for the overall ratings an app receives and may not be representative of what all users believe about the app. However, mobile application developers still need their ratings to be high to be successful.

