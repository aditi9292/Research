While Stowaway is a powerful statical analysis tool which has been used in a substantial amount of previous research~\cite{Pearce:2012:APS:2414456.2414498,Stevens_investigatinguser,jeon2011dr}, it does suffer some drawbacks. Malicious code may be obfuscated and unnecessary API methods inserted into the application, rationalizing the permission~\cite{6698893}. Static analysis techniques can also be hindered by the Java reflection and may lead to inaccuracies~\cite{Sridharan:2006:RCP:1133255.1134027,Tripp:2009:TET:1542476.1542486}. These types of limitations are inherent to all statical analysis tools. While similar reverse engineering techniques have been successfully used in previous works~\cite{Lee_2013,6687155}, no reverse engineering process can ever be expected to be totally accurate. However, based on manually verifying a small subset of our results and the previous research, we have a high confidence in our reverse engineering process.

We only analyzed applications from Google Play and not other sources such as AppksAPK or F-Droid, which would have led to more varied application origins. However, we feel the diversity of our applications was already quite robust since we collected 68,513 applications from 41 genres. We also only examined free applications in our research due to cost constants. Thus, the measurements comparison of apps is not representative of the entire Google Play market. Our results only apply as a comparison of free apps, not with paid apps.
