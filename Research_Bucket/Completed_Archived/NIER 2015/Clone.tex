
%%%%%%%%%%%%%%%%%%%%%%%%%%%%%%

\subsection{Code Clones}


%Many previous works have stated that code clones are undesirable since they often lead to more bugs and make their remediation process more difficult and expensive~\cite{Mondal:2012:ESC:2387358.2387360,Duala-Ekoko:2010:CRD:1767751.1767754,Baker:1995:FDN:832303.836911,Baxter:1998:CDU:850947.853341}. Clones may also substantially raise the maintenance costs associated with an application~\cite{Juergens:2009:CCM:1555001.1555062}, the importance of which is highlighted by the fact that the maintenance phase of a project has been found to encompass between 40\% and 90\% of the totsl cost of a software project~\cite{Shukla:2008:ESM:1342211.1342232,Ducasse:1999:LIA:519621.853389,SMR:SMR225,Ueda:2002:GMS:823457.824039,Boehm:2001:SDR:619059.621640,Erlikh:2000:LLS:612986.613032}. Ultimately, unintentionally making inconsistently applied bug fixes to cloned code across a software system increases the likeliness of further system faults~\cite{Deissenboeck_2010}.

We define the four types of code clones using the definitions from Roy et al.~\cite{Roy:2009:CEC:1530898.1531101}. Type-1 clones are the simplest, representing identical code except for variations in whitespace, comments, and layout. Type-2 clones have variations in identifiers, types, whitespace, literals, layout, and comments, but are otherwise syntactically identical. Type-3 clones are fragments which are copied and have modifications such as added or removed statements, variations in literals, identifiers, whitespace, layout and comments. Type-4 clones, the most difficult to detect, are code segments that perform the same computation, but have been implemented using different syntactic variants.

To assist software practitioners in detecting and managing code clones, clone detection tools have been indispensable in detecting clone-related bugs and even security vulnerabilities in software systems~\cite{Dang:2012:XTC:2420950.2421004}. Of the numerous clone detection tools, many have been able to detect the simpler clones: type-1, type-2, and type-3. To the best of our knowledge, only two techniques MeCC~\cite{Kim:2011:MMC:1985793.1985835} and CCCD~\cite{wcre2013} are able to detect the most difficult kind of clone, type-4s.


In order to demonstrate.... we created two



Concolic analysis combines concrete and symbolic values in order to traverse all possible paths (up to a given length) of an app. Traditionally it has been used for software testing~\cite{Sen:2005:CCU:1081706.1081750}, code clone detection~\cite{krutz2013cccd}, and vulnerability recognition~\cite{Chen:2014:CIB:2554850.2554875}. Since concolic analysis is not affected by syntax or comments, identically traversed paths are indications of duplicate functionality, and therefore functionally equivalent code~\cite{krutz2013cccd,krutz2013code}. These traversed paths are expressed in the form of~\emph{concolic output} which represents the execution path tree and typically displays the utilized path conditions and representative input variables.


Two type-4 clone examples examples of heartbeat are shown in Table~\ref{table:heartbeedexample}.

\noindent
\begin{table*}
\centering
\begin{tabular}{c | c}
\bfseries xxxxxxxxxxxx \#1  & \bfseries xxxxxxxxxxxx \\ \hline \hline
\begin{lstlisting}
boolean shouldBeRunning=true;
int smallInterval=10;
long lastHeartbeat=0;
int heartbeatInterval=10;
while (shouldBeRunning){
  Thread.sleep(smallInterval);
  if(System.currentTimeMillis()-lastHeartbeat>
    heartbeatInterval){
    sendHeartbeat();
    lastHeartbeat= System.currentTimeMillis();
  }
}
\end{lstlisting}
&
\begin{lstlisting}
long lastRunTime=0;	
long timeSpan=System.currentTimeMillis();
long timeSinceLastRun=
    System.currentTimeMillis()-lastRunTime;
  if(timeSinceLastRun>10) {
    sendHeartbeat();
	lastRunTime = System.currentTimeMillis();
}
\end{lstlisting}

\end{tabular}
\caption{An Example of Type-2 clones from Roy~\label{table:heartbeedexample}}
\end{table*}


We then ran concolic analysis on these two code segments which produced the matching concolic output shown in Table~\ref{fig:exampleoutput}.

\noindent
\begin{table}[h] %h for here, t for top, b for bottom
\caption{Diff of Type-4 Clone Concolic Output}
~\label{table:concolicoutputcomparision}
\centering
\begin{tabular}{ p{3.8cm} | p{3.8cm} }
\multicolumn{1}{c}{\textbf{Concolic Segment \#1}} & \multicolumn{1}{c}{\textbf{Concolic Segment \#2}} \\ \hline \hline
\begin{lstlisting}[style=ConcolicOutput]
### PCs: 1 1 0
a_1_SYMINT,
a_1_SYMINT,d1_2_SYMREAL,
a_1_SYMINT,d1_2_SYMREAL,s1_3_SYMSTRING,
\end{lstlisting}
&
\begin{lstlisting}[style=ConcolicOutput]
### PCs: 1 1 0
a_1_SYMINT,
a_1_SYMINT,d1_2_SYMREAL,
a_1_SYMINT,d1_2_SYMREAL,s1_3_SYMSTRING,
\end{lstlisting}

\end{tabular}
\label{fig:exampleoutput}
\end{table}





%%%%%%%%%%%%%%%%%%%%%%%%%%%%%%%

